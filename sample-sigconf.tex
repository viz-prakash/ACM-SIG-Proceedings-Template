% Comments from ACM template.  { curly braces is used for folding} -Vijay
% Template is from https://www.acm.org/publications/proceedings-template
% Read this: https://www.acm.org/publications/taps/latex-best-practices

% Accepted packages by ACM
% https://authors.acm.org/proceedings/production-information/accepted-latex-packages

% Subfigure example: https://latex-tutorial.com/subfigure-latex/

% Clean before submission to archive:
% https://github.com/google-research/arxiv-latex-cleaner
% Use command: python -m arxiv_latex_cleaner tex_proj_dir

%{
%%
%% This is file `sample-sigconf.tex',
%% generated with the docstrip utility.
%%
%% The original source files were:
%%
%% samples.dtx  (with options: `sigconf')
%% 
%% IMPORTANT NOTICE:
%% 
%% For the copyright see the source file.
%% 
%% Any modified versions of this file must be renamed
%% with new filenames distinct from sample-sigconf.tex.
%% 
%% For distribution of the original source see the terms
%% for copying and modification in the file samples.dtx.
%% 
%% This generated file may be distributed as long as the
%% original source files, as listed above, are part of the
%% same distribution. (The sources need not necessarily be
%% in the same archive or directory.)
%%
%% Commands for TeXCount
%TC:macro \cite [option:text,text]
%TC:macro \citep [option:text,text]
%TC:macro \citet [option:text,text]
%TC:envir table 0 1
%TC:envir table* 0 1
%TC:envir tabular [ignore] word
%TC:envir displaymath 0 word
%TC:envir math 0 word
%TC:envir comment 0 0
%%
%%
%}
%% The first command in your LaTeX source must be the \documentclass command.
%\documentclass[sigconf]{acmart}
% anonymous hides the author names & acknowledgement,
% even if they are mentioned -Vijay
\documentclass[sigconf,anonymous]{acmart} 
\settopmatter{printacmref=false}
% Removes the ACM reference format on the first page -Vijay
\renewcommand\footnotetextcopyrightpermission[1]{}
% removes the ACM copyright permission on first page -Vijay
% \pagestyle{plain} %removes running headers,
% works only if put after /maketitle -Vijay

% Comments from ACM template -Vijay
%{
%% NOTE that a single column version is required for 
%% submission and peer review. This can be done by changing
%% the \doucmentclass[...]{acmart} in this template to 
%% \documentclass[manuscript,screen]{acmart}
%% 
%% To ensure 100% compatibility, please check the white list of
%% approved LaTeX packages to be used with the Master Article Template at
%% https://www.acm.org/publications/taps/whitelist-of-latex-packages 
%% before creating your document. The white list page provides 
%% information on how to submit additional LaTeX packages for 
%% review and adoption.
%% Fonts used in the template cannot be substituted; margin 
%% adjustments are not allowed.
%}

% BibTeX commands -Vijay
%{
%%
%% \BibTeX command to typeset BibTeX logo in the docs
\AtBeginDocument{%
  \providecommand\BibTeX{

% ACM copyright commands -Vijay
%{
%% Rights management information.  This information is sent to you
%% when you complete the rights form.  These commands have SAMPLE
%% values in them; it is your responsibility as an author to replace
%% the commands and values with those provided to you when you
%% complete the rights form.
\setcopyright{acmcopyright}
\copyrightyear{2018}
\acmYear{2018}
\acmDOI{XXXXXXX.XXXXXXX}
%}

% ACM proceedings commands -Vijay
%{
%% These commands are for a PROCEEDINGS abstract or paper.
\acmConference[Conference acronym 'XX]{Make sure to enter the correct
  conference title from your rights confirmation emai}{June 03--05,
  2018}{Woodstock, NY}
%
%  Uncomment \acmBooktitle if th title of the proceedings is different
%  from ``Proceedings of ...''!
%
%\acmBooktitle{Woodstock '18: ACM Symposium on Neural Gaze Detection,
%  June 03--05, 2018, Woodstock, NY} 
\acmPrice{15.00}
\acmISBN{978-1-4503-XXXX-X/18/06}
%}

% comments from original template -Vijay
%{
%%
%% Submission ID.
%% Use this when submitting an article to a sponsored event. You'll
%% receive a unique submission ID from the organizers
%% of the event, and this ID should be used as the parameter to this command.
%%\acmSubmissionID{123-A56-BU3}

%%
%% For managing citations, it is recommended to use bibliography
%% files in BibTeX format.
%%
%% You can then either use BibTeX with the ACM-Reference-Format style,
%% or BibLaTeX with the acmnumeric or acmauthoryear sytles, that include
%% support for advanced citation of software artefact from the
%% biblatex-software package, also separately available on CTAN.
%%
%% Look at the sample-*-biblatex.tex files for templates showcasing
%% the biblatex styles.
%%

%%
%% The majority of ACM publications use numbered citations and
%% references.  The command \citestyle{authoryear} switches to the
%% "author year" style.
%%
%% If you are preparing content for an event
%% sponsored by ACM SIGGRAPH, you must use the "author year" style of
%% citations and references.
%% Uncommenting
%% the next command will enable that style.
%%\citestyle{acmauthoryear}

%}
%%
%% end of the preamble, start of the body of the document source.
\begin{document}

%%
%% The "title" command has an optional parameter,
%% allowing the author to define a "short title" to be used in page headers.
\title{The Name of the Title is Hope changed}

% Author names -Vijay
%{ 

%%
%% The "author" command and its associated commands are used to define
%% the authors and their affiliations.
%% Of note is the shared affiliation of the first two authors, and the
%% "authornote" and "authornotemark" commands
%% used to denote shared contribution to the research.
\author{Ben Trovato}
\authornote{Both authors contributed equally to this research.}
\email{trovato@corporation.com}
\orcid{1234-5678-9012}
\author{G.K.M. Tobin}
\authornotemark[1]
\email{webmaster@marysville-ohio.com}
\affiliation{%
  \institution{Institute for Clarity in Documentation}
  \streetaddress{P.O. Box 1212}
  \city{Dublin}
  \state{Ohio}
  \country{USA}
  \postcode{43017-6221}
}

\author{Lars Th{\o}rv{\"a}ld}
\affiliation{%
  \institution{The Th{\o}rv{\"a}ld Group}
  \streetaddress{1 Th{\o}rv{\"a}ld Circle}
  \city{Hekla}
  \country{Iceland}}
\email{larst@affiliation.org}

\author{Valerie B\'eranger}
\affiliation{%
  \institution{Inria Paris-Rocquencourt}
  \city{Rocquencourt}
  \country{France}
}

\author{Aparna Patel}
\affiliation{%
 \institution{Rajiv Gandhi University}
 \streetaddress{Rono-Hills}
 \city{Doimukh}
 \state{Arunachal Pradesh}
 \country{India}}

\author{Huifen Chan}
\affiliation{%
  \institution{Tsinghua University}
  \streetaddress{30 Shuangqing Rd}
  \city{Haidian Qu}
  \state{Beijing Shi}
  \country{China}}

\author{Charles Palmer}
\affiliation{%
  \institution{Palmer Research Laboratories}
  \streetaddress{8600 Datapoint Drive}
  \city{San Antonio}
  \state{Texas}
  \country{USA}
  \postcode{78229}}
\email{cpalmer@prl.com}

\author{John Smith}
\affiliation{%
  \institution{The Th{\o}rv{\"a}ld Group}
  \streetaddress{1 Th{\o}rv{\"a}ld Circle}
  \city{Hekla}
  \country{Iceland}}
\email{jsmith@affiliation.org}

\author{Julius P. Kumquat}
\affiliation{%
  \institution{The Kumquat Consortium}
  \city{New York}
  \country{USA}}
\email{jpkumquat@consortium.net}

%%
%% By default, the full list of authors will be used in the page
%% headers. Often, this list is too long, and will overlap
%% other information printed in the page headers. This command allows
%% the author to define a more concise list
%% of authors' names for this purpose.
\renewcommand{\shortauthors}{Trovato and Tobin, et al.}

%}

\input{sections/acm-template-sections/abstract} %abstract is moved to abstract.tex

% Put ACM Computing Classification System from 
% https://dl.acm.org/ccs; not necessary (confirm with conference)
%{
\ccsdesc[500]{Computer systems organization~Embedded systems}
\ccsdesc[300]{Computer systems organization~Redundancy}
\ccsdesc{Computer systems organization~Robotics}
\ccsdesc[100]{Networks~Network reliability}


%%
%% Keywords. The author(s) should pick words that accurately describe
%% the work being presented. Separate the keywords with commas.
\keywords{datasets, neural networks, gaze detection, text tagging}
%}

%% A "teaser" image appears between the author and affiliation
%% information and the body of the document, and typically spans the
%% page.
\begin{teaserfigure}
  \includegraphics[width=\textwidth]{sections/acm-template-sections/images/sampleteaser}
  \caption{Seattle Mariners at Spring Training, 2010.}
  \Description{Enjoying the baseball game from the third-base
  seats. Ichiro Suzuki preparing to bat.}
  \label{fig:teaser}
\end{teaserfigure}

%%
%% This command processes the author and affiliation and title
%% information and builds the first part of the formatted document.
\maketitle
\pagestyle{plain}
\settopmatter{printfolios=true} % adds page numbers -Vijay
% add comments in color code -Vijay
\newcommand{\XXX}[0]{{\color{red} XXX}}
\newcommand{\danny}[1]{{\color{blue} #1}}
\newcommand{\vijay}[1]{{\color{purple} {#1}{-- Vijay}}}


\input{sections/acm-template-sections/introduction}

\input{sections/acm-template-sections/template-overview}

\input{sections/acm-template-sections/modifications}

\input{sections/acm-template-sections/typefaces}

\input{sections/acm-template-sections/title-information}

\input{sections/acm-template-sections/authors-and-affiliations}

\input{sections/acm-template-sections/rights-information}

\input{sections/acm-template-sections/ccs-concepts-and-keywords}

\section{Sectioning Commands}

Your work should use standard \LaTeX\ sectioning commands:
\verb|section|, \verb|subsection|, \verb|subsubsection|, and
\verb|paragraph|. They should be numbered; do not remove the numbering
from the commands.

Simulating a sectioning command by setting the first word or words of
a paragraph in boldface or italicized text is {\bfseries not allowed.}


\input{sections/acm-template-sections/tables}

\input{sections/acm-template-sections/math-equations}

\section{Figures}

The ``\verb|figure|'' environment should be used for figures. One or
more images can be placed within a figure. If your figure contains
third-party material, you must clearly identify it as such, as shown
in the example below.
\begin{figure}[h]
  \centering
  \includegraphics[width=\linewidth]{sections/acm-template-sections/images/sample-franklin}
  \caption{1907 Franklin Model D roadster. Photograph by Harris \&
    Ewing, Inc. [Public domain], via Wikimedia
    Commons. (\url{https://goo.gl/VLCRBB}).}
  \Description{A woman and a girl in white dresses sit in an open car.}
\end{figure}

Your figures should contain a caption which describes the figure to
the reader.

Figure captions are placed {\itshape below} the figure.

Every figure should also have a figure description unless it is purely
decorative. These descriptions convey what’s in the image to someone
who cannot see it. They are also used by search engine crawlers for
indexing images, and when images cannot be loaded.

A figure description must be unformatted plain text less than 2000
characters long (including spaces).  {\bfseries Figure descriptions
  should not repeat the figure caption – their purpose is to capture
  important information that is not already provided in the caption or
  the main text of the paper.} For figures that convey important and
complex new information, a short text description may not be
adequate. More complex alternative descriptions can be placed in an
appendix and referenced in a short figure description. For example,
provide a data table capturing the information in a bar chart, or a
structured list representing a graph.  For additional information
regarding how best to write figure descriptions and why doing this is
so important, please see
\url{https://www.acm.org/publications/taps/describing-figures/}.

\subsection{Figures showing how to crop and position} 

Here is original figure fitted to \verb|width=linewidth| and \verb|scaled| to 0.50, without any crop in the Figure \ref{fig:diffie-hellman-illus}.
\begin{figure}[!h]
    \centering
    \includegraphics[width=\linewidth,scale=0.50]{sections/acm-template-sections/images/Diffie-Hellman_Key_Exchange.svg.png}
    \caption{Illustration of Diffie-Hellman key exchange. (\url{https://upload.wikimedia.org/wikipedia/commons/4/46/Diffie-Hellman_Key_Exchange.svg})}
    \label{fig:diffie-hellman-illus}
\end{figure}

To crop image use \verb|trim={0 0 0 0.5cm},clip| where trim length are mentioned by sides mentioned as \textit{left, bottom, right, top} respectively to \verb|trim|, and \verb|clip| activates clipping. See the figure \ref{fig:diffie-hellman-illus-2}. Here environment \begin{verbatim}
    \begin{figure}[floats here]
    \end{figure}
\end{verbatim}
is used instead of
\begin{verbatim}
    \begin{figure*}[floats here]
    \end{figure*}
\end{verbatim}. To read more about floats \href{https://www.overleaf.com/learn/latex/Errors/\%60!h\%27_float_specifier_changed_to_\%60!ht\%27}{see}. 

\begin{figure}[!h]
    \centering
    \includegraphics[width=\linewidth,trim={0 0.5 0 0.5cm},clip,scale=0.50]{sections/acm-template-sections/images/Diffie-Hellman_Key_Exchange.svg.png}
    \caption{Illustration of Diffie-Hellman key exchange with clipped image from top and bottom. (\url{https://upload.wikimedia.org/wikipedia/commons/4/46/Diffie-Hellman_Key_Exchange.svg})}
    \label{fig:diffie-hellman-illus-2}
\end{figure}


\subsection{The ``Teaser Figure''}

A ``teaser figure'' is an image, or set of images in one figure, that
are placed after all author and affiliation information, and before
the body of the article, spanning the page. If you wish to have such a
figure in your article, place the command immediately before the
\verb|\maketitle| command:
\begin{verbatim}
  \begin{teaserfigure}
    \includegraphics[width=\textwidth]{sampleteaser}
    \caption{figure caption}
    \Description{figure description}
  \end{teaserfigure}
\end{verbatim}


\input{sections/acm-template-sections/citations-and-bibliographies}

\input{sections/acm-template-sections/acknowledgements}

\input{sections/acm-template-sections/appendices}

\input{sections/acm-template-sections/multi-language-papers}

\section{SIGCHI Extended Abstracts}

The ``\verb|sigchi-a|'' template style (available only in \LaTeX\ and
not in Word) produces a landscape-orientation formatted article, with
a wide left margin. Three environments are available for use with the
``\verb|sigchi-a|'' template style, and produce formatted output in
the margin:
\begin{itemize}
\item {\verb|sidebar|}:  Place formatted text in the margin.
\item {\verb|marginfigure|}: Place a figure in the margin.
\item {\verb|margintable|}: Place a table in the margin.
\end{itemize}


%%
%% The acknowledgments section is defined using the "acks" environment
%% (and NOT an unnumbered section). This ensures the proper
%% identification of the section in the article metadata, and the
%% consistent spelling of the heading.
\begin{acks}
To Robert, for the bagels and explaining CMYK and color spaces.
\end{acks}

%%
%% The next two lines define the bibliography style to be used, and
%% the bibliography file.
\bibliographystyle{ACM-Reference-Format}
\bibliography{sample-base}

%%
%% If your work has an appendix, this is the place to put it.
\appendix

\section{Research Methods}

\subsection{Part One}

Lorem ipsum dolor sit amet, consectetur adipiscing elit. Morbi
malesuada, quam in pulvinar varius, metus nunc fermentum urna, id
sollicitudin purus odio sit amet enim. Aliquam ullamcorper eu ipsum
vel mollis. Curabitur quis dictum nisl. Phasellus vel semper risus, et
lacinia dolor. Integer ultricies commodo sem nec semper.

\subsection{Part Two}

Etiam commodo feugiat nisl pulvinar pellentesque. Etiam auctor sodales
ligula, non varius nibh pulvinar semper. Suspendisse nec lectus non
ipsum convallis congue hendrerit vitae sapien. Donec at laoreet
eros. Vivamus non purus placerat, scelerisque diam eu, cursus
ante. Etiam aliquam tortor auctor efficitur mattis.

\section{Online Resources}

Nam id fermentum dui. Suspendisse sagittis tortor a nulla mollis, in
pulvinar ex pretium. Sed interdum orci quis metus euismod, et sagittis
enim maximus. Vestibulum gravida massa ut felis suscipit
congue. Quisque mattis elit a risus ultrices commodo venenatis eget
dui. Etiam sagittis eleifend elementum.

Nam interdum magna at lectus dignissim, ac dignissim lorem
rhoncus. Maecenas eu arcu ac neque placerat aliquam. Nunc pulvinar
massa et mattis lacinia.

\section{End of the appendix}
Always include figures and table of full width of both column like below with \begin{verbatim}
    \begin{figure*}
        \centering
        \includegraphics{}
        \caption{Caption}
        \label{fig:my_label}
    \end{figure}
\end{verbatim}, otherwise images will not be rendered.

\begin{figure*}[!hb]
    \centering
    %\includegraphics[trim={0.2cm 0 5.3cm 3.6cm},clip,scale=0.32]{images/category-wise-distr/curl_vendor_categorization.pdf}
    \includegraphics[trim={0 0 0 0.5cm},clip,scale=0.20]{sections/acm-template-sections/images/Hyperbola_one_over_x.svg.png}
    
    \caption{Explanation of division-by-zero error from wiki. (\url{https://upload.wikimedia.org/wikipedia/commons/4/43/Hyperbola_one_over_x.svg})}
    \label{fig:div-zero}
\end{figure*}


\end{document}
\endinput
%%
%% End of file `sample-sigconf.tex'.
