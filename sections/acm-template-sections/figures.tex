\section{Figures}

The ``\verb|figure|'' environment should be used for figures. One or
more images can be placed within a figure. If your figure contains
third-party material, you must clearly identify it as such, as shown
in the example below.
\begin{figure}[h]
  \centering
  \includegraphics[width=\linewidth]{sections/acm-template-sections/images/sample-franklin}
  \caption{1907 Franklin Model D roadster. Photograph by Harris \&
    Ewing, Inc. [Public domain], via Wikimedia
    Commons. (\url{https://goo.gl/VLCRBB}).}
  \Description{A woman and a girl in white dresses sit in an open car.}
\end{figure}

Your figures should contain a caption which describes the figure to
the reader.

Figure captions are placed {\itshape below} the figure.

Every figure should also have a figure description unless it is purely
decorative. These descriptions convey what’s in the image to someone
who cannot see it. They are also used by search engine crawlers for
indexing images, and when images cannot be loaded.

A figure description must be unformatted plain text less than 2000
characters long (including spaces).  {\bfseries Figure descriptions
  should not repeat the figure caption – their purpose is to capture
  important information that is not already provided in the caption or
  the main text of the paper.} For figures that convey important and
complex new information, a short text description may not be
adequate. More complex alternative descriptions can be placed in an
appendix and referenced in a short figure description. For example,
provide a data table capturing the information in a bar chart, or a
structured list representing a graph.  For additional information
regarding how best to write figure descriptions and why doing this is
so important, please see
\url{https://www.acm.org/publications/taps/describing-figures/}.

\subsection{Figures showing how to crop and position} 

Here is original figure fitted to \verb|width=linewidth| and \verb|scaled| to 0.50, without any crop in the Figure \ref{fig:diffie-hellman-illus}.
\begin{figure}[!h]
    \centering
    \includegraphics[width=\linewidth,scale=0.50]{sections/acm-template-sections/images/Diffie-Hellman_Key_Exchange.svg.png}
    \caption{Illustration of Diffie-Hellman key exchange. (\url{https://upload.wikimedia.org/wikipedia/commons/4/46/Diffie-Hellman_Key_Exchange.svg})}
    \label{fig:diffie-hellman-illus}
\end{figure}

To crop image use \verb|trim={0 0 0 0.5cm},clip| where trim length are mentioned by sides mentioned as \textit{left, bottom, right, top} respectively to \verb|trim|, and \verb|clip| activates clipping. See the figure \ref{fig:diffie-hellman-illus-2}. Here environment \begin{verbatim}
    \begin{figure}[floats here]
    \end{figure}
\end{verbatim}
is used instead of
\begin{verbatim}
    \begin{figure*}[floats here]
    \end{figure*}
\end{verbatim}. To read more about floats \href{https://www.overleaf.com/learn/latex/Errors/\%60!h\%27_float_specifier_changed_to_\%60!ht\%27}{see}. 

\begin{figure}[!h]
    \centering
    \includegraphics[width=\linewidth,trim={0 0.5 0 0.5cm},clip,scale=0.50]{sections/acm-template-sections/images/Diffie-Hellman_Key_Exchange.svg.png}
    \caption{Illustration of Diffie-Hellman key exchange with clipped image from top and bottom. (\url{https://upload.wikimedia.org/wikipedia/commons/4/46/Diffie-Hellman_Key_Exchange.svg})}
    \label{fig:diffie-hellman-illus-2}
\end{figure}


\subsection{The ``Teaser Figure''}

A ``teaser figure'' is an image, or set of images in one figure, that
are placed after all author and affiliation information, and before
the body of the article, spanning the page. If you wish to have such a
figure in your article, place the command immediately before the
\verb|\maketitle| command:
\begin{verbatim}
  \begin{teaserfigure}
    \includegraphics[width=\textwidth]{sampleteaser}
    \caption{figure caption}
    \Description{figure description}
  \end{teaserfigure}
\end{verbatim}
